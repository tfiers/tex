% Star indicate that these are short commands (not containing whole paragraphs)
\newcommand*{\captionn}[2]{\caption{\textbf{#1}. #2}}
\newcommand*{\chapterr}[2]{\chapter[#1]{#1\\{\large #2}}}
\newcommand*{\sectionn}[2]{\section[#1]{#1\\{\large #2}}}
\newcommand*{\subsectionn}[2]{\subsection[#1]{#1\\{\large #2}}}

\newenvironment{deflist}
    {\vspace{1em}\tabularx{\textwidth}{l X}}
    {\endtabularx}
\newcommand*{\notation}[2]{$#1$ & #2 \\[1em]}
\newcommand*{\abbr}[3]{\textbf{#1} & \emph{#2}. #3 \\[1em]}
\newcommand*{\gloss}[2]{\textbf{#1} & #2 \\[1em]}

% Default arguments, from the guide: http://tex.loria.fr/ctan-doc/macros/latex/doc/html/usrguide/node18.html
\newcommand*{\img}[2][1]{\includegraphics[width=#1\textwidth]{#2}}

\newcommand*{\reals}{\mathbb{R}}
\newcommand*{\trans}{^T}  % More straight: ^{\mkern-1.5mu\mathsf{T}}
\newcommand*{\tdots}{\,..\,}  % two dots
\newcommand*{\tcdots}{\,\cdot\cdot\,}

\DeclareMathOperator*{\argmax}{arg\,max}
\DeclareMathOperator*{\argmin}{arg\,min}
\DeclareMathOperator*{\Var}{Var}
\DeclareMathOperator*{\Cov}{Cov}
\DeclareMathOperator*{\st}{subject\ to\ }

\newcommand*{\mat}[1]{\vb{#1}}
\newcommand*{\diag}[1]{\operatorname{diag}\qty(#1)}
\newcommand*{\qed}{\hfill\ensuremath{\square}}
\newcommand*{\eps}{\varepsilon}
\newcommand*{\had}{\mathrel{\mathsmaller{\mathsmaller{\odot}}}}
% Pointwise multiplication (Hadamard product).
% From https://latex.org/forum/viewtopic.php?t=27766
