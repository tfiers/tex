% Stars indicate short commands (not containing whole paragraphs)

\newcommand*{\TOC}{\pdfbookmark{Contents}{contents-bookmark}{\hypersetup{hidelinks}\tableofcontents*}}
\newcommand*{\References}{\clearpage\printbibliography[title=References]}

% Adds an unnumbered \chapter* to the table of contents (using an arbitrary ID as argument).
\newcommand*{\addToTOC}[1]{\addcontentsline{toc}{chapter}{\nameref{#1}}}

% Subtitles for headings
\newcommand*{\chapterr}[2]{\chapter[#1]{#1\\{\large #2}}}
\newcommand*{\sectionn}[2]{\section[#1]{#1\\{\large #2}}}
\newcommand*{\subsectionn}[2]{\subsection[#1]{#1\\{\large #2}}}


\newcommand*{\singlechapter}{
    % To have "fig/table/eq 1" instead of "fig/table/eq 1.1".
    \counterwithout{figure}{chapter}
    \counterwithout{table}{chapter}
    \counterwithout{equation}{chapter}

    % To have section numbers "1" instead of "0.1".
    \renewcommand*\thesection{\arabic{section}}
}

% An image with a label overlaid ("A" e.g.).
% Usage: \panelimg[rel-width][x,y]{filename}{Panel-letter}
%   where (x,y) is the label position, given as fractions of image width and
%   height, relative to the top-left corner.
\NewDocumentCommand{\panelimg}{O{0.5} O{0.1, 0} m m}{\begin{tikzpicture}
        \node[anchor = north west, inner sep = 0] (image) at (0,0) {\includegraphics[width=#1\textwidth]{#3}};
        \begin{scope}[x={(image.north east)},y={(image.south west)}]
            \node at (#2) {\textbf{\sffamily\Large{#4}}};
        \end{scope}
    \end{tikzpicture}}

% Reference-able subfigures.
% Usage: \subimg[rel-width]{filename}{label}
\NewDocumentCommand{\subimg}{O{0.6} m m}{\begin{subfigure}{#1\textwidth}
        \includegraphics{#2}\caption{}\label{#3}
    \end{subfigure}}

\newcommand*{\panelref}[2]{\cref{#1}#2}
\newcommand*{\Panelref}[2]{\Cref{#1}#2}

% Usage: \captionn[sep]{Bolded figure title}{Rest of figure caption}
% Weird: using \\ inside captions errors, but \newline works.
\NewDocumentCommand{\captionn}{O{ } m m}{\caption{\textbf{#2}#1#3}}
\NewDocumentCommand{\captionnoftable}{O{ } m m}{\captionof{table}{\textbf{#2}#1#3}}

% For use in captions
\newcommand*{\Left}{\emph{Left}\xspace}
\newcommand*{\Center}{\emph{Center}\xspace}
\newcommand*{\Right}{\emph{Right}\xspace}
\newcommand*{\Top}{\emph{Top}\xspace}
\newcommand*{\Middle}{\emph{Middle}\xspace}
\newcommand*{\Bottom}{\emph{Bottom}\xspace}
\newcommand*{\panel}[1]{\newline\textbf{#1.}}


% Add star to list all authors. Default = et al.
\NewDocumentCommand\authoryear{s m}{%
    \IfBooleanTF#1%
    {\citeauthor{#2} \citeyear{#2}}%  star
    {\citeauthor*{#2} \citeyear{#2}}%  no star
}

\NewDocumentCommand\citefull{s m}{%
    \IfBooleanTF#1%
    {\authoryear*{#2} \cite{#2}}%  star
    {\authoryear{#2} \cite{#2}}%  no star
}

\NewDocumentCommand\Authorref{s m}{%
    \IfBooleanTF#1%
    {\citeauthor{#2} \cite{#2}}%  star
    {\citeauthor*{#2} \cite{#2}}%  no star
}

\newcommand{\range}[3]{
    \SIrange[range-phrase=\,--\,,range-units=single]{#1}{#2}{#3}}

% We have to set column widths manually (instead of using `\tabularx{\linewidth}{l X} ... \endtabularx` to auto-scale the second column). This is because ltablex (the tabularx equivalent of longtable) does not work with `newenvironment'.
\newenvironment{deflist}
{\vspace{0.8em}\begin{longtable}{p{4em} p{32em}}}
        {\end{longtable}}

\newcommand*{\notation}[2]{$#1$ & #2 \\[0.6em]}
\newcommand*{\abbr}[3]{\textbf{#1} & #2. #3 \\[0.6em]}
\newcommand*{\gloss}[2]{\textbf{#1} & #2 \\[0.6em]}

\newcommand*{\fnm}{\footnotemark{}\xspace}
\newcommand*{\fnt}[1]{\footnotetext{#1}}

\newcommand*{\reals}{\mathbb{R}}
\newcommand*{\trans}{^T}  % More straight: ^{\mkern-1.5mu\mathsf{T}}
\newcommand*{\tdots}{\,..\,}  % two dots
\newcommand*{\tcdots}{\,\cdot\cdot\,}

\DeclareMathOperator*{\argmax}{arg\,max\,}
\DeclareMathOperator*{\argmin}{arg\,min\,}
\newcommand*{\mean}[1]{\overline{#1}}

\newcommand*{\mat}[1]{\vb{#1}}
\newcommand*{\diag}[1]{\operatorname{diag}\qty(#1)}
\newcommand*{\qed}{\hfill\ensuremath{\square}}
\newcommand*{\eps}{\varepsilon}
\newcommand*{\had}{\mathrel{\mathsmaller{\mathsmaller{\odot}}}}
% Pointwise multiplication (Hadamard product).
% From https://latex.org/forum/viewtopic.php?t=27766
\newcommand*{\given}{\;\vert\;}
