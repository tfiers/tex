% Stars indicate short commands (not containing whole paragraphs)

% Assuming organization: ./modules/[Chapter-Name]/index.tex
\newcommand*{\module}[1]{\import{modules/#1/}{index}}

\newcommand*{\TOC}{\pdfbookmark{Contents}{contents-bookmark}{\hypersetup{hidelinks}\tableofcontents*}}
\newcommand*{\References}{\clearpage\printbibliography[title=References]}

% Adds an unnumered \chapter* to the table of contents (using an arbitrary ID as argument).
\newcommand*{\addToTOC}[1]{\addcontentsline{toc}{chapter}{\nameref{#1}}}

% Subtitles for headings
\newcommand*{\chapterr}[2]{\chapter[#1]{#1\\{\large #2}}}
\newcommand*{\sectionn}[2]{\section[#1]{#1\\{\large #2}}}
\newcommand*{\subsectionn}[2]{\subsection[#1]{#1\\{\large #2}}}

% Page number goals for headings
\newcommand*{\chapterp}[2]{\chapter{#2 [#1]}}
\newcommand*{\sectionp}[2]{\section{#2 [#1]}}
\newcommand*{\subsectionp}[2]{\subsection{#2 [#1]}}

% >>> Convention: images are stored in a `figures/' subdir <<<
%
% Usage: \img[rel-width]{filename}
\NewDocumentCommand{\img}{O{1} m}{\includegraphics[width=#1\textwidth]{figures/#2}}

% An image with a label overlaid ("A" e.g.).
% Usage: \panelimg[rel-width][x,y]{filename}{Panel-letter}
%   where (x,y) is the label position, given as fractions of image width and
%   height, relative to the top-left corner.
\NewDocumentCommand{\panelimg}{O{0.5} O{0.1, 0} m m}{\begin{tikzpicture}
\node[anchor = north west, inner sep = 0] (image) at (0,0) {\img[#1]{#3}};
\begin{scope}[x={(image.north east)},y={(image.south west)}]
    \node at (#2) {\textbf{\sffamily\Large{#4}}};
\end{scope}
\end{tikzpicture}}

% Reference-able subfigures.
% Usage: \subimg[rel-width]{filename}{label}
\NewDocumentCommand{\subimg}{O{0.6} m m}{\begin{subfigure}{#1\textwidth}
\img[1]{#2}\caption{}\label{#3}
\end{subfigure}}

\newcommand*{\panelref}[2]{\cref{#1}#2}
\newcommand*{\Panelref}[2]{\Cref{#1}#2}

% Usage: \captionn[sep]{Bolded figure title}{Rest of figure caption}
% Weird: using \\ inside captions errors, but \newline works.
\NewDocumentCommand{\captionn}{O{. } m m}{\caption{\textbf{#2}#1#3}}
\NewDocumentCommand{\captionnoftable}{O{. } m m}{\captionof{table}{\textbf{#2}#1#3}}

% For use in captions
\newcommand*{\Left}{\emph{Left}}
\newcommand*{\Center}{\emph{Center}}
\newcommand*{\Right}{\emph{Right}}
\newcommand*{\Top}{\emph{Top}}
\newcommand*{\Middle}{\emph{Middle}}
\newcommand*{\Bottom}{\emph{Bottom}}
\newcommand*{\panel}[1]{\newline\textbf{#1.}}

\newcommand*{\citefull}[1]{\citeauthor*{#1} \citeyear{#1} \cite{#1}}

\newcommand{\range}[3]{
\SIrange[range-phrase=\,--\,,range-units=single]{#1}{#2}{#3}}

\newenvironment{deflist}
    {\vspace{1em}\tabularx{\textwidth}{l X}}
    {\endtabularx}
\newcommand*{\notation}[2]{$#1$ & #2 \\[0.6em]}
\newcommand*{\abbr}[3]{\textbf{#1} & #2. #3 \\[0.6em]}
\newcommand*{\gloss}[2]{\textbf{#1} & #2 \\[0.6em]}

\newcommand*{\reals}{\mathbb{R}}
\newcommand*{\trans}{^T}  % More straight: ^{\mkern-1.5mu\mathsf{T}}
\newcommand*{\tdots}{\,..\,}  % two dots
\newcommand*{\tcdots}{\,\cdot\cdot\,}

\DeclareMathOperator*{\argmax}{arg\,max\,}
\DeclareMathOperator*{\argmin}{arg\,min\,}
\DeclareMathOperator*{\Var}{Var}
\DeclareMathOperator*{\Cov}{Cov}
\DeclareMathOperator*{\st}{subject\ to\ }
\DeclareMathOperator*{\sigmoid}{\sigma}

\newcommand*{\mat}[1]{\vb{#1}}
\newcommand*{\diag}[1]{\operatorname{diag}\qty(#1)}
\newcommand*{\qed}{\hfill\ensuremath{\square}}
\newcommand*{\eps}{\varepsilon}
\newcommand*{\had}{\mathrel{\mathsmaller{\mathsmaller{\odot}}}}
% Pointwise multiplication (Hadamard product).
% From https://latex.org/forum/viewtopic.php?t=27766
