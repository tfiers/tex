% Star indicate that these are short commands (not containing whole paragraphs)
\newcommand*{\TOC}{\pdfbookmark{Contents}{contents-bookmark}\tableofcontents*}
\newcommand*{\References}{\clearpage\printbibliography[title=References]}

% Assuming organization: ./parts/[Chapter-Name]/index.tex
\newcommand*{\includepart}[1]{\import{parts/#1/}{index}}

% Subtitles for headings
\newcommand*{\chapterr}[2]{\chapter[#1]{#1\\{\large #2}}}
\newcommand*{\sectionn}[2]{\section[#1]{#1\\{\large #2}}}
\newcommand*{\subsectionn}[2]{\subsection[#1]{#1\\{\large #2}}}

% Page number goals for headings
\newcommand*{\chapterp}[2]{\chapter{#2 [#1]}}
\newcommand*{\sectionp}[2]{\section{#2 [#1]}}
\newcommand*{\subsectionp}[2]{\subsection{#2 [#1]}}

% Convention: images are stored in a `figures/' subdir
\newcommand*{\img}[2][1]{\includegraphics[width=#1\textwidth]{figures/#2}}

% Usage: \captionn{Bolded figure title}{Rest of figure caption}
\newcommand*{\captionn}[2]{\caption{\textbf{#1}. #2}}
\newcommand*{\captionnoftable}[2]{\captionof{table}{\textbf{#1}. #2}}

% For use in captions
\newcommand*{\Left}{\emph{Left}}
\newcommand*{\Center}{\emph{Center}}
\newcommand*{\Right}{\emph{Right}}
\newcommand*{\Top}{\emph{Top}}
\newcommand*{\Middle}{\emph{Middle}}
\newcommand*{\Bottom}{\emph{Bottom}}

\newcommand*{\citefull}[1]{\citeauthor*{#1} \citeyear{#1} \cite{#1}}

\newcommand{\range}[3]{
\SIrange[range-phrase=\,--\,,range-units=single]{#1}{#2}{#3}}

\newenvironment{deflist}
    {\vspace{1em}\tabularx{\textwidth}{l X}}
    {\endtabularx}
\newcommand*{\notation}[2]{$#1$ & #2 \\[0.6em]}
\newcommand*{\abbr}[3]{\textbf{#1} & \emph{#2}. #3 \\[0.6em]}
\newcommand*{\gloss}[2]{\textbf{#1} & #2 \\[0.6em]}

\newcommand*{\reals}{\mathbb{R}}
\newcommand*{\trans}{^T}  % More straight: ^{\mkern-1.5mu\mathsf{T}}
\newcommand*{\tdots}{\,..\,}  % two dots
\newcommand*{\tcdots}{\,\cdot\cdot\,}

\DeclareMathOperator*{\argmax}{arg\,max\,}
\DeclareMathOperator*{\argmin}{arg\,min\,}
\DeclareMathOperator*{\Var}{Var}
\DeclareMathOperator*{\Cov}{Cov}
\DeclareMathOperator*{\st}{subject\ to\ }

\newcommand*{\mat}[1]{\vb{#1}}
\newcommand*{\diag}[1]{\operatorname{diag}\qty(#1)}
\newcommand*{\qed}{\hfill\ensuremath{\square}}
\newcommand*{\eps}{\varepsilon}
\newcommand*{\had}{\mathrel{\mathsmaller{\mathsmaller{\odot}}}}
% Pointwise multiplication (Hadamard product).
% From https://latex.org/forum/viewtopic.php?t=27766
