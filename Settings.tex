% We're assuming memoir here.
%
% If you want to write an article instead of someting larger with chapters,
% use the `article' option for documentclass, and specify
% `chapterstyle{komalike}'

% Package loading times could be profiled as here:
% https://tex.stackexchange.com/questions/141661/package-loading-speeds
% There, tikz takes 0.1 seconds, siunitx 0.2 seconds.

% For package documentation links: hover over pkg name in VS Code w/ Latex-Workshop
% (If no local docs, as with the TinyTeX distribution, use CTAN link).

% Use 8 bits per glyph in the output, instead of 7.
% Advantages: https://tex.stackexchange.com/a/677
\usepackage[T1]{fontenc}

% Allow UTF-8 characters in .tex files
\usepackage[utf8]{inputenc}

% Filter warnings
\usepackage{silence}
% Second arg is start of warning message
\WarningFilter{caption}{Unused \captionsetup[subfloat]}
\WarningFilter{microtype}{Unable to apply patch `footnote'}

\usepackage{pdfpages}
% To be able to include modern pdf's
\pdfminorversion=8

\headstyles{komalike}
\chapterstyle{bianchi}

% cft = Memoir mnemonic for (table of) Contents, (list of) Figures,
% and (list of) tables
\renewcommand{\cftchapterfont}{\bfseries\sffamily}
\renewcommand{\cftsectionfont}{\normalfont\sffamily}

% `\includegraphics` and its options
\usepackage{graphicx}
% Specify root directories for your figures with:
% \graphicspath{{subdir1/}{subdir2/}}

\usepackage{tikz}

% `\printinunitsof{mm}\prntlen{\textwidth}`
% `\uselengthunit{mm}\printlength{\textwidth}'
\usepackage{printlen}

% Provides \begin{subfigure}
\usepackage{subcaption}

% On figure placement:
%  - https://tex.stackexchange.com/a/39020
%  - https://robjhyndman.com/hyndsight/latex-floats/
%
% The default placement specifier is (probably) [tbp],
% for top, bottom, own page.

% Palatino (a nice serif font) for serif text & for math.
% \usepackage[sc]{mathpazo}
%
% Serif text: Palatino
% Sans-serif text: Computer Modern sans-serif
% Math text: Default (Computer Modern)
\usepackage{palatino}
\renewcommand{\sfdefault}{cmss}

% Make typography easier on the eye.
% (Characters may expand into the margins so that the margins are psycho-visually straight; Font width may vary line by line, so that the interword spacing becomes more even).
\usepackage{microtype}
% For this to work with mathpazo, you have to manually install cm-super, and run updmap.

\usepackage{caption}
\captionsetup{margin=15pt}
\captionsetup{font={sf}}
\captionsetup[subfloat]{labelfont={bf}, labelformat=simple, labelsep=period}

\DeclareCaptionLabelFormat{bold}{\textbf{#2.}}
\captionsetup{subrefformat=bold}
% \captionsetup[subrefformat]{labelfont={bf}, labelformat=simple, labelsep=period}

\setfloatadjustment{figure}{\sffamily\centerfloat}
\setfloatadjustment{table}{\sffamily\centerfloat}

% To place a table and figure next to each other (\captionof{table}{..}):
% https://tex.stackexchange.com/questions/103206/figure-and-table-next-to-each-other-memoir-class
\usepackage{capt-of}

% Nicer table lines. Use \toprule, \midrule, \bottomrule
% (Has tabularx as an automatic dependency)
\usepackage{booktabs}
%
% To additionally remove left and right space:
% \begin{tabular}{@{}lll@{}}

\usepackage{longtable}

% Combines tabularx with longtables. No extra commands necessary for multi-page tables.
\usepackage{ltablex}

% Some more space between rows
\newcommand{\ra}[1]{\renewcommand{\arraystretch}{#1}}
\ra{1.2}

% Use as \multirow{nrows}{*}{Cell contents}
\usepackage{multirow}

% A command \makecell to use line breaks in tables
\usepackage{makecell}

% For \mathrel{\mathsmaller{.}} e.g.
\usepackage{relsize}

% For equations: alignment, numbering, `\text{}`, ...
\usepackage{amsmath}

% For ℝ e.g.
\usepackage{amssymb}

% A command \bm{.} to boldface math symbols or entire math expressions.
\usepackage{bm}

% \num{1e-10} becomes nice exponential notation.
% Also: \SI{1e-10}{\meter\per\second}
\usepackage{siunitx}
\sisetup{detect-all}  % Adapt to surrounding font
\sisetup{inter-unit-product = \ensuremath{{} \cdot {}}}
\sisetup{per-mode = symbol}  % C / s
% From {1 Hz to 5 Hz} to {1 -- 5 Hz}
% \sisetup{range-phrase = \,--\,}
% \sisetup{range-units = single}

% \usepackage{physics}  -- no: https://tex.stackexchange.com/a/628184/153868
%
% When using with siunitx, we get the warning:
%     "Detected the "physics" package:
%      omitting definition of \qty.
%      If you want to use \qty with the siunitx definition, add
%      \AtBeginDocument{\RenewCommandCopy\qty\SI}
%      to your preamble."
% Haven't succeeded on \WarningFilter'ing it, and the proposed fix doesn't do it either.
%
% We used physics for nice derivative symbols etc; we'll define those ourselves instead.
% See:
% - https://ctan.mirror.garr.it/mirrors/ctan/macros/latex/contrib/physics/physics.pdf)
%    ↪ `\flatfrac` is from physics. Just use `/'.
% - https://tex.stackexchange.com/questions/225523/how-to-write-partial-differential-equation-ex-dq-dt-ds-dt-with-real-partial-d
% - `\mathrm{d}` :)


% Ions
% \usepackage[version=4]{mhchem}
% I'll try and use normal supscripts for ions, to save a dep.

%
\usepackage{geometry}
\geometry{
    top=25mm,
    bottom=35mm,
}

% To rotate a page. \begin{landscape}, \end{landscape}
\usepackage{pdflscape}

% Paragraph spacing
\setlength{\parindent}{0em}
\setlength{\parskip}{1.2em}

% https://tex.stackexchange.com/a/26522
\setlength{\textfloatsep}{22.0pt plus 2.0pt minus 4.0pt}
\setlength{\floatsep}{18.0pt plus 2.0pt minus 2.0pt}

\setlength{\abovecaptionskip}{8pt plus 3pt minus 2pt}

% Subsection spacing
\setbeforesecskip{5.0ex plus 1ex minus .2ex}
\setbeforesubsecskip{4.5ex plus 1ex minus .2ex}

% Allows to call commands as `\cmd' instead of `\cmd{}', while still inserting a
% space after. https://tex.stackexchange.com/a/31092/153868
\usepackage{xspace}


% `biblatex' is fresher than `natbib'. "biber" is the program, biblatex is
% the package.
\usepackage[
    style=alphabetic,
    % sorting=none,  % sort in citation order
    maxbibnames=20,
]{biblatex}
% Note: Biblatex options can only be changed in \usepackage, so here.
%
% In your preamble:
% \addbibresource{myrefs.bib}
% '.bib' extension is necessary.

% Don't print url if there's doi.
% Thx https://tex.stackexchange.com/a/154875/153868
\DeclareSourcemap{
    \maps[datatype=bibtex]{
        \map{
            \step[fieldsource=doi,final]
            \step[fieldset=url,null]
        }
    }
}

% Linebreaks in urls.
\usepackage{xurl}

% Better commands (macros / functions)
\usepackage{xparse}

% `\vref` cleverly adds the page number to a reference.
\usepackage{varioref}


\usepackage[hidelinks]{hyperref}
\usepackage{bookmark}
% ↪ Gets rid of a hyperref warning.
%      https://tex.stackexchange.com/a/167952/153868

\usepackage[
    noabbrev,       % "figure 1" instead of "fig. 1".
    nameinlink,   % makes "figure 1" clickable, not just the "1".
]{cleveref}

\definecolor{PleasantBlue}{RGB}{13, 127, 172}
\hypersetup{
    colorlinks = true,
    linkcolor = PleasantBlue,
    citecolor = PleasantBlue,
    urlcolor = PleasantBlue,
}

% for the \singlechapter command.
\usepackage{chngcntr}
